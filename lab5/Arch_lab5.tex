\documentclass[a4paper,14pt]{article}
\usepackage[utf8]{inputenc}
\usepackage{amssymb,amsfonts,amsmath,cite,enumerate,float,indentfirst}
\usepackage[english, russian]{babel}
\usepackage[dvips]{graphicx}
\usepackage{multirow}
\graphicspath{{images/}}
\usepackage{geometry} % Меняем поля страницы
\geometry{left=2cm}% левое поле
\geometry{right=2cm}% правое поле
\geometry{top=2cm}% верхнее поле
\geometry{bottom=2cm}% нижнее поле

\begin{document}
\begin{titlepage}
\newpage

\begin{center}
Министерство науки и высшего образования Российской Федерации\\
% \vspace{3em}
Федеральное государственное автономное образовательное учреждение высшего образования\\
«Национальный исследовательский университет ИТМО»\\
Факультет информационных технологий и программирования\\
\end{center}

\vspace{\fill}
% \vspace{28em}

\begin{center}
\textbf{Лабораторная работа №5}\\
\textit{Исследование работы ЭВМ при асинхронном обмене данными с ВУ.}
\end{center}

\vspace{\fill}
% \vspace{20em}

\newbox{\lbox}
\savebox{\lbox}{\hbox{Шайдулин Михаил Андреевич}}
\newlength{\maxl}
\setlength{\maxl}{\wd\lbox}
\hfill\parbox{14cm}{
\hspace*{5cm}Выполнил студент:\hfill\hbox to\maxl{Шайдулин Михаил Андреевич \hfill}\\
\hspace*{5cm}Группа:\hfill\hbox to\maxl{M3106}\\
}


\vspace{8em}
% \vspace{\fill}

\begin{center}
Санкт-Петербург \\2021
\end{center}

\end{titlepage}


\section*{Цель работы:}
Изучение организации системы ввода-вывода базовой ЭВМ, команд ввода-вывода и исследование процесса функционирования ЭВМ при обмене данными по сигналам готовности внешних устройств.

\section*{Ход работы:}

\section {Заданное слово и коды его символов:}
\begin{tabular}{|l|l|l|l|l|l|l|l|}
    \hline
    \textbf{Символ} & \textbf{К} & \textbf{Р} & \textbf{Е} & \textbf{М} & \textbf{Е} & \textbf{Н} & \textbf{(-)Ь}\\
    \hline
    Код & EC & F2 & E5 & ED & E5 & EE & F8\\
    \hline
\end{tabular}

\section {Текст программы}
\begin{tabular}{|c|c|c|l|}
\hline
\textbf{Адрес} & \textbf{Код команды} & \textbf{Мнемоника} & \textbf{Комментарий} \\\hline
01F & 000C & M & Ссылка на элементы массива \\
010 & E101 & TSF 1 & Проверка флага 1, если = 1, то ввод \\
011 & C010 & BR 10 & Иначе повтор \\
012 & E201 & IN & Ввод \\
013 & E001 & CLF 1 & Отчистка флага 1 \\
014 & 381F & MOV M & Перемещение в память элемента массива \\
015 & 0020 & ISZ K & Увеличение на 1 счётчика элементов массива \\
016 & C010 & BR 10 &  \\
017 & F000 & HLT & Остановка \\
. . . & . . . &  &  \\
020 & FFFA & K & Отрицательное количество элементов массива \\
021 & 0000 &  & 1-й элемент массива \\
022 & 0000 &  & 2-й элемент массива \\
023 & 0000 &  & 3-й элемент массива \\
024 & 0000 &  & 4-й элемент массива \\
025 & 0000 &  & 5-й элемент массива \\
026 & 0000 &  & 6-й элемент массива \\
\hline
\end{tabular}


\section{Таблица трассировки}
\begin{tabular}{llllllllll}
    \hline
    \multicolumn{2}{l}{
    \multirow{1}{10em}{\textbf{Выполняемая команда}}} &
    \multicolumn{6}{l}{
    \multirow{1}{15em}{\textbf{Содержимое регистров после выполнения команды}}} &
    \multicolumn{2}{l}{
    \multirow{1}{11em}{\textbf{Ячейка, содержимое которой изменилось после выполнения команды}}}\\
    \\\\\\
    \hline
    Адрес & Код & CK & PA & РК & РД & А & C & Адрес & Новый код\\
    010 & E101 & 012 & 01 & E101 & E101 & 0000 & 0\\
    012 & E201 & 013 & 01 & E201 & E201 & 00EC & 0\\
    013 & E001 & 014 & 01 & E001 & E001 & 00EC & 0\\
    014 & 381F & 015 & 01F & 381F & 00EC & 00EC & 0 & 021 & 00EC\\
    015 & 0020 & 016 & 020 & 0020 & FFFA & 00EC & 0 & 020 & FFFB\\
    016 & C010 & 010 & 010 & C010 & C010 & 00EC & 0\\
    010 & E101 & 012 & 01 & E101 & E101 & 00EC & 0\\
    012 & E201 & 013 & 01 & E201 & E201 & 00F2 & 0\\
    013 & E001 & 014 & 01 & E001 & E001 & 00F2 & 0\\
    014 & 381F & 015 & 01F & 381F & 00F2 & 00F2 & 0 & 022 & 00F2\\
    015 & 0020 & 016 & 020 & 0020 & FFFB & 00F2 & 0 & 020 & FFFC\\
    016 & C010 & 010 & 010 & C010 & C010 & 00F2 & 0\\
    010 & E101 & 012 & 01 & E101 & E101 & 00F2 & 0\\
    012 & E201 & 013 & 01 & E201 & E201 & 00E5 & 0\\
    013 & E001 & 014 & 01 & E001 & E001 & 00E5 & 0\\
    014 & 381F & 015 & 01F & 381F & 00E5 & 00E5 & 0 & 023 & 00E5\\
    015 & 0020 & 016 & 020 & 0020 & FFFC & 00E5 & 0 & 020 & FFFD\\
    016 & C010 & 010 & 010 & C010 & C010 & 00E5 & 0\\
    010 & E101 & 012 & 01 & E101 & E101 & 00E5 & 0\\
    012 & E201 & 013 & 01 & E201 & E201 & 00ED & 0\\
    013 & E001 & 014 & 01 & E001 & E001 & 00ED & 0\\
    014 & 381F & 015 & 01F & 381F & 00ED & 00ED & 0 & 024 & 00ED\\
    015 & 0020 & 016 & 020 & 0020 & FFFD & 00ED & 0 & 020 & FFFE\\
    016 & C010 & 010 & 010 & C010 & C010 & 00ED & 0\\
    010 & E101 & 012 & 01 & E101 & E101 & 00ED & 0\\
    012 & E201 & 013 & 01 & E201 & E201 & 00E5 & 0\\
    013 & E001 & 014 & 01 & E001 & E001 & 00E5 & 0\\
    014 & 381F & 015 & 01F & 381F & 00E5 & 00E5 & 0 & 025 & 00E5\\
    015 & 0020 & 016 & 020 & 0020 & FFFD & 00E5 & 0 & 020 & FFFF\\
    016 & C010 & 010 & 010 & C010 & C010 & 00E5 & 0\\
    010 & E101 & 012 & 01 & E101 & E101 & 00E5 & 0\\
    012 & E201 & 013 & 01 & E201 & E201 & 00EE & 0\\
    013 & E001 & 014 & 01 & E001 & E001 & 00EE & 0\\
    014 & 381F & 015 & 01F & 381F & 00EE & 00EE & 0 & 026 & 00EE\\
    015 & 0020 & 017 & 020 & 0020 & FFFF & 00EE & 0 & 020 & 0000\\
    017 & F000 & 018 & 017 & F000 & F000 & 00EE & 0\\
    \hline
\end{tabular}

\section{Описание программы}
\begin{itemize}
    \item Программа считывает коды символов слова и записывает их в массив.
    \item Ячейки 010 – 017 – область выполнения программы, 020 - 026 – область данных.
    \item Ячейки 010 и 017 содержат первую и последнюю выполняемые команды программы.
\end{itemize}

\section{Цикл выполнения программы}
\begin{enumerate}
    \item Проверка флага ВУ-1.
    \item Если флаг = 1, то ввод, иначе п1.
    \item Отчистка флага.
    \item Перемещение элемента массива в память.
    \item Увеличение на 1 счётчика элементов. Если он = 0, то остановка, иначе п1.
\end{enumerate}


\end{document}
