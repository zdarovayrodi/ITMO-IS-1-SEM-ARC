\documentclass[a4paper,14pt]{article}
\usepackage[utf8]{inputenc}
\usepackage{amssymb,amsfonts,amsmath,cite,enumerate,float,indentfirst}
\usepackage[english, russian]{babel}
\usepackage[dvips]{graphicx}
\usepackage{multirow}
\graphicspath{{images/}}
\usepackage{geometry} % Меняем поля страницы
\geometry{left=2cm}% левое поле
\geometry{right=2cm}% правое поле
\geometry{top=2cm}% верхнее поле
\geometry{bottom=2cm}% нижнее поле

\begin{document}
\begin{titlepage}
\newpage

\begin{center}
Министерство науки и высшего образования Российской Федерации\\
% \vspace{3em}
Федеральное государственное автономное образовательное учреждение высшего образования\\
«Национальный исследовательский университет ИТМО»\\
Факультет информационных технологий и программирования\\
\end{center}

\vspace{\fill}
% \vspace{28em}

\begin{center}
\textbf{Лабораторная работа №3}\\
\textit{Исследование работы ЭВМ при выполнении циклических программ.}
\end{center}

\vspace{\fill}
% \vspace{20em}

\newbox{\lbox}
\savebox{\lbox}{\hbox{Шайдулин Михаил Андреевич}}
\newlength{\maxl}
\setlength{\maxl}{\wd\lbox}
\hfill\parbox{14cm}{
\hspace*{5cm}Выполнил студент:\hfill\hbox to\maxl{Шайдулин Михаил Андреевич \hfill}\\
\hspace*{5cm}Группа:\hfill\hbox to\maxl{M3106}\\
}


\vspace{8em}
% \vspace{\fill}

\begin{center}
Санкт-Петербург \\2021
\end{center}

\end{titlepage}


\section*{Цель работы:}
Изучение способов организации циклических программ и
исследование порядка функционирования ЭВМ при выполнении циклических программ.
\section*{Ход работы:}

\section{Восстановить текст программы}

\begin{tabular}{|c|c|l|l|}
\hline
\textbf{Адрес} & \textbf{Код команды} & \textbf{Мнемоника} & \textbf{Комментарий} \\\hline
00A & 0010 & M & Ссылка на элемент массива \\
00B & 0000 &  &  \\
00C & 0000 &  &  \\
00D & 0000 &  &  \\
00E & 0000 &  &  \\
00F & 0000 &  &  \\
010 & 8080 &  & 1-й элемент массива \\
011 & ABDA &  & 2-й элемент массива \\
012 & 630D &  & 3-й элемент массива \\
013 & 71B0 &  & 4-й элемент массива \\
014 & FFFC & K & Отрицательное кол-во элементов \\
015 & 0000 & R & Результат \\
016 & \textbf{F200} & CLA & Очистка аккумулятора \\
017 & 480A & ADD M & (M) + (A) $\to$ A \\
018 & A01A &  &  \\
019 & 2045 &  &  \\
01A & 0014 &  &  \\
01B & C016 &  &  \\
01C & F000 &  &  \\
01D & 0000 &  &  \\
. . . &  &  &  \\
045 & 0000 &  &  \\
046 & F200 &  &  \\
047 & F800 &  &  \\
048 & 4015 &  &  \\
049 & 3015 &  &  \\
04А & C845 &  &  \\

. . .







	M









К
R
CLA
ADD (M)
BMI 1A
JSR 45
ISZ K
BR 16
HLT

. . .

CLA
INC
ADD R
MOV R
BR 45	Ссылка на элемент массива



Если (А) < 0, то переход к 01А
(СК) $\to$ 045, 045 + 1 $\to$ CK (подпрограмма)
Если K < 0, то (K) + 1 $\to$ K, иначе (СК) + 1 $\to$ CK
16 $\to$ CK
Остановка

. . .
Адрес возврата из подпрограммы
Очистка аккумулятора
(А) + 1 $\to$ A
(R) + (A) $\to$ A
(A) $\to$ R
Переход к 045

\hline
\end{tabular}

\section{Таблица трассировки}
\begin{tabular}{ll|l|l|l|l|l|l|l|l}
    \hline
    \multicolumn{2}{l}{
    \multirow{1}{10em}{\textbf{Выполняемая команда}}} &
    \multicolumn{6}{l}{
    \multirow{1}{15em}{\textbf{Содержимое регистров после выполнения команды}}} &
    \multicolumn{2}{l}{
    \multirow{1}{11em}{\textbf{Ячейка, содержимое которой изменилось после выполнения команды}}}\\
    \\\\\\
    \hline
    Адрес & Код & CK & PA & РК & РД & А & C & Адрес & Новый код\\
013 & F200 & 014 & 013 & F200 & F200 & 0000 & 0 &  & \\
014 & 480E & 015 & 00E & 480E & 0378 & 0378 & 0\\
015 & B018 & 016 & 018 & B018 & 0378 & 0378 & 0\\
016 & 4011 & 017 & 011 & 4011 & 0000 & 0378 & 0\\
017 & 3011 & 018 & 011 & 3011 & 0000 & 0378 & 0 & 011 & 0378\\
018 & 0012 & 019 & 012 & 0012 & FFFC & 0378 & 0 & 012 & FFFD\\
019 & C013 & 013 & 013 & C013 & 0378 & 0378 & 0\\
013 & F200 & 014 & 013 & F200 & F200 & 0000 & 0\\
014 & 480E & 015 & 00E & 480E & 0000 & 0000 & 0\\
015 & B018 & 018 & 018 & B018 & 0000 & 0000 & 0\\
018 & 0012 & 019 & 012 & 0012 & FFFD & 0000 & 0 & 012 & FFFE\\
019 & C013 & 013 & 013 & C013 & 0000 & 0000 & 0\\
013 & F200 & 014 & 013 & F200 & F200 & 0000 & 0\\
014 & 480E & 015 & 00E & 480E & F0EB & F0EB & 0\\
015 & B018 & 016 & 018 & B018 & F0EB & F0EB & 0\\
016 & 4011 & 017 & 011 & 4011 & 0378 & F463 & 0\\
017 & 3011 & 018 & 011 & 3011 & F463 & F463 & 0 & 011 & F463\\
018 & 0012 & 019 & 012 & 0012 & FFFE & F463 & 0 & 012 & FFFF\\
019 & C013 & 013 & 013 & C013 & F463 & F463 & 0\\
013 & F200 & 014 & 013 & F200 & F200 & 0000 & 0\\
014 & 480E & 015 & 00E & 480E & 0377 & 0377 & 0\\
015 & B018 & 016 & 018 & B018 & 0377 & 0377 & 0\\
016 & 4011 & 017 & 011 & 4011 & F463 & F7DA & 0\\
017 & 3011 & 018 & 011 & 3011 & F7DA & F7DA & 0 & 011 & F7DA\\
018 & 0012 & 20 & 012 & 0012 & FFFF & F7DA & 0\\
020 & F000 & 21 & 020 & F000 & F000 & F7DA & 0\\
\hline
\end{tabular}

\section{Описание программы}
Программа находит сумму элементов массива. Если элемент не равен 0, то значение $R$ увеличивается на значение данного элемента. С каждой итерацией $К$ увеличивается на 1. Когда К станет = 0, программа завершится.\\

\noindent
\textbf{013-01А} – область выполнения программы\\
\textbf{00Е, 01В-01Е} – область данных\\
\textbf{011} - результат\\
\textbf{013, 01F} содержат первую и последнюю выполняемые команды программы

\end{document}
