\documentclass[a4paper,14pt]{article}
\usepackage[utf8]{inputenc}
\usepackage{amssymb,amsfonts,amsmath,cite,enumerate,float,indentfirst}
\usepackage[english, russian]{babel}
\usepackage[dvips]{graphicx}
\usepackage{multirow}
\graphicspath{{images/}}
\usepackage{geometry} % Меняем поля страницы
\geometry{left=2cm}% левое поле
\geometry{right=2cm}% правое поле
\geometry{top=2cm}% верхнее поле
\geometry{bottom=2cm}% нижнее поле

\begin{document}
\begin{titlepage}
\newpage

\begin{center}
Министерство науки и высшего образования Российской Федерации\\
% \vspace{3em}
Федеральное государственное автономное образовательное учреждение высшего образования\\
«Национальный исследовательский университет ИТМО»\\
Факультет информационных технологий и программирования\\
\end{center}

\vspace{\fill}
% \vspace{28em}

\begin{center}
\textbf{Домашнее задание №3}\\
\textit{Программирование обмена данными с внешними устройствами}
\end{center}

\vspace{\fill}
% \vspace{20em}

\newbox{\lbox}
\savebox{\lbox}{\hbox{Шайдулин Михаил Андреевич}}
\newlength{\maxl}
\setlength{\maxl}{\wd\lbox}
\hfill\parbox{14cm}{
\hspace*{5cm}Выполнил студент:\hfill\hbox to\maxl{Шайдулин Михаил Андреевич \hfill}\\
\hspace*{5cm}Группа:\hfill\hbox to\maxl{M3106}\\
}


\vspace{8em}
% \vspace{\fill}

\begin{center}
Санкт-Петербург \\2021
\end{center}

\end{titlepage}


\section*{Цель работы:}
Написать комплекс программ, обеспечивающий обмен данными с ВУ в
режиме прерывания программы. Основная программа должна наращивать на 1
(начиная с 0) содержимое (обозначим его буквой Х) какой-либо ячейки памяти. Цикл
для наращивания Х не должен содержать более трех команд. Вывод всегда
осуществляется на ВУ-3 в асинхронном режиме. Выводится только восемь младших
разрядов результата.\\
\textit{По запросу ВУ-1 вывести -2Х+5, а по запросу ВУ-2 вывести 3Х/4.}\\
\indent
Составить методику проверки правильности выполнения разработанного
комплекса на базовой ЭВМ, т.е. написать последовательность действий оператора
(пользователя) базовой ЭВМ, которые необходимо выполнить, чтобы проверить все
возможные режимы работы комплекса программ (при появлении запроса
прерывания от любого ВУ) и получить заданное количество результатов.

\section*{Ход работы:}

\section{Код программы}
\begin{tabular}{|c|c|c|l|}
\hline
\textbf{Адрес} & \textbf{Код команды} & \textbf{Мнемоника} & \textbf{Комментарий} \\\hline

    000 & & & Адрес возврата к основной программе\\
001& С030 & BR 30 & Переход к основному тексту подпрограммы\\
. . . & . . . & . . . & . . .\\
020& FA00 & EI & Разрешение прерывания\\
021& F200 & CLA & Очистка аккумулятора\\
022& F800 & INC & \\
023& 30 & MOV X & Цикл наращивания Х на 1\\
024& C022 & BR 22 & \\
025& 0000 &  & \\
. . . & . . . & . . . & . . .\\
030& 305A & MOV 5C & Сохранение содержимого аккумулятора и регистра переноса\\
031& F600 & ROL & \\
032& 305C & MOV 5D & \\
& & &\\
033& E101 & TSF 1 & Проверка флага 1\\
034& C036 & BR 36 & Если флаг 1 = 1, то переход к 036\\
035& C039 & BR 39 & Иначе к 039\\
036& E102 & TSF 2 & Проверка флага 2\\
037& C054 & BR 54 & Если флаг 2 = 1, то переход к 054\\
038& C047 & BR 47 & Иначе к 047\\
& & &\\
039& F200 & CLA & \\
03A& 605C & SUB 5C & -Х\\
03B& 605C & SUB 5C & -2X\\
03C& F800 & INC & -2X + 1\\
03D& F800 & INC & -2X + 2\\
03E& F800 & INC & -2X + 3\\
03F& F800 & INC & -2X + 4\\
040& F800 & INC & -2X + 5\\
041& E001 & CLF 1 & Отчистка флага 1\\
042& E103 & TSF 3 & Если флаг 3 готов к выводу, то вывод\\
043& C042 & BR 42 & Иначе повторная проверка готовности\\
044& E303 & OUT & Вывод\\
045& E003 & CLF 3 & Отчистка флага 3\\
046& C055 & BR 55 & Переход к 55\\
& & &\\
047& F200 & CLA & \\
048& 405C & ADD 5C & X\\
049& 405C & ADD 5C & 2X\\
04A& 405C & ADD 5C & 3X\\
04B& F700 & ROR & 3X \backslash 2\\
04C& F700 & ROR & 3X \backslash 4\\
04E& E002 & CLF 2 & Отчистка флага 2\\
04F& E103 & TSF 3 & Если флаг 3 = 1, то вывод\\
050& C04F & BR 4F & Иначе повторная проверка готовности\\
051& E303 & OUT & Вывод\\
052& E003 & CLF 3 & Отчистка флага 3\\
053& C055 & BR 55 & Переход к 55\\
& & &\\
054& E003 & CLF 3 & Отчистка флага 3\\
& & &\\
055& F200 & CLA & \\
056& 405D & ADD 5D & \\
057& F700 & ROR & Возврат значений аккумулятора и РС\\
058& F200 & CLA & \\
059& 405C & ADD 5C & \\
05A& FA00 & EI & Разрешение прерывания\\
    \hline
\end{tabular}

\newpage
\begin{tabular}{|c|c|c|l|}
\hline
\textbf{Адрес} & \textbf{Код команды} & \textbf{Мнемоника} & \textbf{Комментарий} \\\hline
    05B& C800 & BR (0) & Продолжение выполнения основной программы\\
    & & &\\
    05C& 0000 & X & Аккумулятор\\
    05D& 0000 & C & Регистр переноса\\
\hline
\end{tabular}

\section {Методика проверки}

\begin{enumerate}
    \item Загрузить комплекс программ в память базовой ЭВМ.
    \item Запустить основную программу в автоматическом режиме с адреса 020.
    \item Установить "Готовность ВУ-1".
    \item После сброса "Готовность ВУ-1" установить "Готовность ВУ-3".
    \item После сброса "Готовность ВУ-3", что означает, что результат выражения -2х+5 был выведен на ВУ-3, установить "Готовность ВУ-2".
    \item После сброса "Готовность ВУ-2" установить "Готовность ВУ-3".
    \item После сброса "Готовность ВУ-3", что означает, что результат выражения 3х/4 был выведен на ВУ-3, установить "Готовность ВУ-3".
    \item После сброса "Готовность ВУ-2" ничего выведено и введено не будет.

\end{enumerate}
\end{document}
