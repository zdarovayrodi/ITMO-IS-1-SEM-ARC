\documentclass[a4paper,14pt]{article}
\usepackage[utf8]{inputenc}
\usepackage{amssymb,amsfonts,amsmath,cite,enumerate,float,indentfirst}
\usepackage[english, russian]{babel}
\usepackage[dvips]{graphicx}
\usepackage{multirow}
\graphicspath{{images/}}
\usepackage{geometry} % Меняем поля страницы
\geometry{left=2cm}% левое поле
\geometry{right=2cm}% правое поле
\geometry{top=2cm}% верхнее поле
\geometry{bottom=2cm}% нижнее поле

\begin{document}
\begin{titlepage}
\newpage

\begin{center}
Министерство науки и высшего образования Российской Федерации\\
% \vspace{3em}
Федеральное государственное автономное образовательное учреждение высшего образования\\
«Национальный исследовательский университет ИТМО»\\
Факультет информационных технологий и программирования\\
\end{center}

\vspace{\fill}
% \vspace{28em}

\begin{center}
\textbf{Лабораторная работа №\#}\\
\textit{Название лабораторной работы}
\end{center}

\vspace{\fill}
% \vspace{20em}

\newbox{\lbox}
\savebox{\lbox}{\hbox{Шайдулин Михаил Андреевич}}
\newlength{\maxl}
\setlength{\maxl}{\wd\lbox}
\hfill\parbox{14cm}{
\hspace*{5cm}Выполнил студент:\hfill\hbox to\maxl{Шайдулин Михаил Андреевич \hfill}\\
\hspace*{5cm}Группа:\hfill\hbox to\maxl{M3106}\\
}


\vspace{8em}
% \vspace{\fill}

\begin{center}
Санкт-Петербург \\2021
\end{center}

\end{titlepage}


\section*{Цель работы:}
text
\section*{Ход работы:}

\section{Таблица трассировки}
\begin{tabular}{llllllllll}
    \hline
    \multicolumn{2}{l}{
    \multirow{1}{10em}{\textbf{Выполняемая команда}}} &
    \multicolumn{6}{l}{
    \multirow{1}{15em}{\textbf{Содержимое регистров после выполнения команды}}} &
    \multicolumn{2}{l}{
    \multirow{1}{11em}{\textbf{Ячейка, содержимое которой изменилось после выполнения команды}}}\\
    \\\\\\
    \hline
    Адрес & Код & CK & PA & РК & РД & А & C & Адрес & Новый код\\
    019 & F200 &  &  &  &  &  &  &  &  \\
    01A & 4016 &  &  &  &  &  &  &  &  \\
    01B & 4017 &  &  &  &  &  &  &  &  \\
    01C & B020 &  &  &  &  &  &  &  &  \\
    01D & F200 &  &  &  &  &  &  &  &  \\
    01E & 3018 &  &  &  &  &  &  &  &  \\
    01F & F000 &  &  &  &  &  &  &  &  \\
    020 & 4016 &  &  &  &  &  &  &  &  \\
    021 & 3018 &  &  &  &  &  &  &  &  \\
    022 & C01F &  &  &  &  &  &  &  &  \\
    \hline
\end{tabular}

\end{document}
