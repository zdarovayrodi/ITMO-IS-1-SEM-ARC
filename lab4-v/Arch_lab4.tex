\documentclass[a4paper,14pt]{article}
\usepackage[utf8]{inputenc}
\usepackage{amssymb,amsfonts,amsmath,cite,enumerate,float,indentfirst}
\usepackage[english, russian]{babel}
\usepackage[dvips]{graphicx}
\usepackage{multirow}
\graphicspath{{images/}}
\usepackage{geometry} % Меняем поля страницы
\geometry{left=2cm}% левое поле
\geometry{right=2cm}% правое поле
\geometry{top=2cm}% верхнее поле
\geometry{bottom=2cm}% нижнее поле

\begin{document}
\begin{titlepage}
\newpage

\begin{center}
Министерство науки и высшего образования Российской Федерации\\
% \vspace{3em}
Федеральное государственное автономное образовательное учреждение высшего образования\\
«Национальный исследовательский университет ИТМО»\\
Факультет информационных технологий и программирования\\
\end{center}

\vspace{\fill}
% \vspace{28em}

\begin{center}
\textbf{Лабораторная работа №4}\\
\textit{Исследование работы ЭВМ при выполнении комплекса программ.}
\end{center}

\vspace{\fill}
% \vspace{20em}

\newbox{\lbox}
\savebox{\lbox}{\hbox{Шайдулин Михаил Андреевич}}
\newlength{\maxl}
\setlength{\maxl}{\wd\lbox}
\hfill\parbox{14cm}{
\hspace*{5cm}Выполнил студент:\hfill\hbox to\maxl{Шайдулин Михаил Андреевич \hfill}\\
\hspace*{5cm}Группа:\hfill\hbox to\maxl{M3106}\\
}


\vspace{8em}
% \vspace{\fill}

\begin{center}
Санкт-Петербург \\2021
\end{center}

\end{titlepage}


\section*{Цель работы:}
Изучение способов связи между программными модулями, команды обращения к подпрограмме и исследование порядка функционирования ЭВМ при выполнении комплекса взаимосвязанных программ.
\section*{Ход работы:}

\section{Текст программы}
\begin{tabular}{|c|c|c|l|}
\hline
\textbf{Адрес} & \textbf{Код команды} & \textbf{Мнемоника} & \textbf{Комментарий} \\\hline
00A & 0010 &&\\
00B & 0000 &&\\
00C & 0000 &&\\
00D & 0000 &&\\
00E & 0000 &&\\
OOF & 0000 &&\\
010 & 8080 &  & данные\\
011 & ABDA &  & данные\\
012 & 630D &  & данные\\
013 & 71B0 &  & данные\\
014 & FFFC &  & счетчик\\
015 & 0000 &  & здесь записывается результат\\
016 & +F200 & CLA & 0 $\to$ A\\
017 & 480A & ADD (00A) & (00A) + (A) $\to$ A\\
018 & A01A & BMI M & если (А) < 0, то 01А $\to$ CK\\
019 & 2045 & JSR M & (CK) $\to$ 045, 045+1 $\to$ CК\\
01A & 0014 & ISZ M & (014) + 1 $\to$ 014, если (014) $\ge$ 0, то (СК) + 1 $\to$ CK\\
01B & C016 & BR M & 016 $\to$ CK\\
01C & F000 & HLT &\\
01D & 0000&&\\
...&...&...&\\
045 & 0000 &&\\
046 & F200 & CLA & 0 $\to$ A\\
047 & F800 & INC & (A) + 1 $\to$ A\\
048 & 4015 & ADD M & (015) + (A) $\to$ A\\
049 & 3015 & MOV M & (A) $\to$ 015\\
04A & C845 & BR (M) & (045) $\to$ CK\\
\hline
\end{tabular}

\section{Таблица трассировки}
\begin{tabular}{llllllllll}
    \hline
    \multicolumn{2}{l}{
    \multirow{1}{10em}{\textbf{Выполняемая команда}}} &
    \multicolumn{6}{l}{
    \multirow{1}{15em}{\textbf{Содержимое регистров после выполнения команды}}} &
    \multicolumn{2}{l}{
    \multirow{1}{11em}{\textbf{Ячейка, содержимое которой изменилось после выполнения команды}}}\\
    \\\\\\
    \hline
    Адрес & Код & CK & PA & РК & РД & А & C & Адрес & Новый код\\
016 &F200 &017 &016 &F200 &F200 &0000 &0 & &\\
017 &480A &018 &010 &480A &8080 &8080 &0\\
018 &A01A &019 &1A &A01A &A01A &8080 &0\\
019 &2045 &046 &045 &2045 &01A &8080 &0\\
046 &F200 &047 &046 &F200 &F200 &0000 &0\\
047 &F800 &048 &047 &F800 &F800 &0001 &0\\
048 &4015 &049 &015 &4015 &0000 &0001 &0\\
049 &3015 &04A &015 &3015 &0001 &0001 &0 &015 &0001\\
04A &C845 &01A &045 &C845 &01A &0001 &0\\
01A &0014 &01B &014 &0014 &FFFC &0001 &0 &014 &FFFD\\
01B &C016 &016 &016 &C016 &C016 &0001 &0\\
016 &F200 &017 &016 &F200 &F2000 &0000 &0\\
017 &480A &018 &011 &480A &ABDA &ABDA &0\\
018 &A01A &019 &1A &A01A &A01A &ABDA &0\\
019 &2045 &046 &045 &2045 &01A &ABDA &0\\
046 &F200 &047 &046 &F200 &F200 &0000 &0\\
047 &F800 &048 &047 &F800 &F800 &0001 &0\\
048 &4015 &049 &015 &4015 &0001 &0002 &0\\
049 &3015 &04A &015 &3015 &0002 &0002 &0 &015 &0002\\
04A &C845 &01A &045 &C845 &01A &0002 &0\\
01A &0014 &01B &014 &0014 &FFFD &0002 &0 &014 &FFFE\\
01B &C016 &016 &016 &C016 &C016 &0002 &0\\
016 &F200 &017 &016 &F200 &ABDA &0000 &0\\
017 &480A &018 &012 &480A &630D &630D &0\\
018 &A01A &019 &1A &A01A &A01A &630D &0\\
019 &2045 &046 &045 &2045 &01A &630D &0\\
046 &F200 &047 &046 &F200 &F200 &0000 &0\\
047 &F800 &048 &047 &F800 &F800 &0001 &0\\
048 &4015 &049 &015 &4015 &0002 &0003 &0\\
049 &3015 &04A &015 &3015 &0002 &0003 &0 &015 &0003\\
04A &C845 &01A &045 &C845 &01A &0003 &0\\
01A &0014 &01B &014 &0014 &FFFE &0003 &0 &014 &FFFF\\
01B &C016 &016 &016 &C016 &C016 &0003 &0\\
016 &F200 &017 &016 &F200 &F2000 &0000 &0\\
017 &480A &018 &013 &480A &71B0 &71B0 &0\\
018 &A01A &019 &1A &A01A &A01A &71B0 &0\\
019 &2045 &046 &045 &2045 &01A &71B0 &0\\
046 &F200 &047 &046 &F200 &F200 &0000 &0\\
047 &F800 &048 &047 &F800 &F800 &0001 &0\\
048 &4015 &049 &015 &4015 &0003 &0004 &0\\
049 &3015 &04A &015 &3015 &0004 &0004 &0 &015 &0004\\
04A &C845 &01A &045 &C845 &01A &0004 &0\\
01A &0014 &01A &014 &0014 &FFFF &0004 &0 &014 &0000\\
01C &F000 &01D &01C &F000 &F000 &0004 &0\\

    \hline
\end{tabular}

\section {Описание программы:}
\begin{itemize}
\item Программа считает неотрицательные элементы массива. Если элемент неотрицателен, то выполняется подпрограмма, которая увеличивает некоторое число R(изначально равное 0) на 1. И так продолжается, пока элементы массива не закончатся.
\item Ячейки 016 – 01D – область выполнения программы, 045 – 04A – подпрограмма, 00А - 0014 – область данных, 015 - результат.
\item Ячейки 016 и 01D содержат первую и последнюю выполняемые команды программы.
\end{itemize}
\end{document}
