\documentclass[a4paper,14pt]{article}
\usepackage[utf8]{inputenc}
\usepackage{amssymb,amsfonts,amsmath,cite,enumerate,float,indentfirst}
\usepackage[english, russian]{babel}
\usepackage[dvips]{graphicx}
\usepackage{multirow}
\graphicspath{{images/}}
\usepackage{geometry} % Меняем поля страницы
\geometry{left=2cm}% левое поле
\geometry{right=2cm}% правое поле
\geometry{top=2cm}% верхнее поле
\geometry{bottom=2cm}% нижнее поле

\begin{document}
\begin{titlepage}
\newpage

\begin{center}
Министерство науки и высшего образования Российской Федерации\\
% \vspace{3em}
Федеральное государственное автономное образовательное учреждение высшего образования\\
«Национальный исследовательский университет ИТМО»\\
Факультет информационных технологий и программирования\\
\end{center}

\vspace{\fill}
% \vspace{28em}

\begin{center}
\textbf{Лабораторная работа №2}\\
\textit{Исследование работы ЭВМ при выполнении разветвляющихся программ.}
\end{center}

\vspace{\fill}
% \vspace{20em}

\newbox{\lbox}
\savebox{\lbox}{\hbox{Шайдулин Михаил Андреевич}}
\newlength{\maxl}
\setlength{\maxl}{\wd\lbox}
\hfill\parbox{14cm}{
\hspace*{5cm}Выполнил студент:\hfill\hbox to\maxl{Шайдулин Михаил Андреевич \hfill}\\
\hspace*{5cm}Группа:\hfill\hbox to\maxl{M3106}\\
}


\vspace{8em}
% \vspace{\fill}

\begin{center}
Санкт-Петербург \\2021
\end{center}

\end{titlepage}


\section*{Цель работы:}
Изучение команд переходов, способов организации
разветвляющихся программ и исследование порядка функционирования ЭВМ при
выполнении таких программ.
\section*{Ход работы:}

\section{Восстановить текст программы}

\begin{tabular}{|c|c|c|l|}
\hline
\textbf{Адрес} & \textbf{Код команды} & \textbf{Мнемоника} & \textbf{Комментарий} \\\hline
016 & CF0B & BR M & M $\to$ CK \\
017 & F0F5 &  &  \\
018 & F000 & HLT &  \\
019 & +F200 & CLA & 0 $\to$ A \\
01A & 4016 & ADD M & (M) + (A) $\to$ A \\
01B & 4017 & ADD M & (M) + (A) $\to$ A \\
01C & B020 & BEQ M & Если (A) и (C) = 0, то M $\to$ CK \\
01D & F200 & CLA & 0 $\to$ A \\
01E & 3018 & MOV M & (A) $\to$ M \\
01F & F000 & HLT &  \\
020 & 4016 & ADD M & (M) + (A) $\to$ A \\
021 & 3018 & MOV M & (A) $\to$ M \\
022 & C01F & BR M & M $\to $ CK \\
023 & 0000 &  &\\
\hline
\end{tabular}

\section{Заполнить таблицу трассировки, выполняя за базовую ЭВМ заданный вариант программы.}
\begin{tabular}{ll|l|l|l|l|l|l|l|l}
    \hline
    \multicolumn{2}{l}{
    \multirow{1}{10em}{\textbf{Выполняемая команда}}} &
    \multicolumn{6}{l}{
    \multirow{1}{15em}{\textbf{Содержимое регистров после выполнения команды}}} &
    \multicolumn{2}{l}{
    \multirow{1}{11em}{\textbf{Ячейка, содержимое которой изменилось после выполнения команды}}}\\
    \\\\\\
    \hline
    Адрес & Код & CK & PA & РК & РД & А & C & Адрес & Новый код\\
    019 & F200 & 01A & 019 & F200 & F200 & 0000 &  &  &  \\
    01A & 4016 & 01B & 016 & 4016 & CF0B & CF0B & 0 &  \\
    01B & 4017 & 01C & 017 & 4017 & F0F5 & C000 & 0 & &\\
    01C & B020 & 01D & 01C & B020 & B020 & C000 & 1 &  &\\
    01D & F200 & 01E & 01D & F200 & F200 & 0000 & 1 &  &\\
    01E & 3018 & 01F & 018 & 3018 & 0000 & 0000 & 1 &  018 & 0000 \\
    01F & F000 & 020 & 01F & F000 & F000 & 0000 & 1 &  &\\
    \hline
\end{tabular}

\section{Описание программы}
Складываем 2 числа. Если сумма равна 0, то записываем результат в ячейку 018 и завершаем программу. В противном случаем обнуляем аккумулятор и заносим в 018 содержимое аккумулятора \textit{\textbf{(CF0B)}}.\\

\noindent
\textbf{019-022} - область выполения команд\\
\textbf{016-017} – область используемых данных\\
\textbf{018} – результат

\end{document}
